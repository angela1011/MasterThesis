% !TeX root = ../main.tex

\chapter{緒論}

\section{前言}
\begingroup
\centering
建立近地表地震速度模型對於地熱儲集層的探勘至關重要,因為它能提供關鍵的地下地質結構資訊,幫助識別潛在地熱熱點,並提高地熱資源評估的準確性。本研究區位於台灣北部的宜蘭平原,其為蘭陽溪沖積作用形成的沖積扇,沉積速率極高,平原呈現近似正三角形狀,每邊約30公里長。宜蘭平原位於沖繩海槽西南端,北西側受雪山山脈、南側受中央山脈所圍限,並受弧後擴張作用影響(圖1),可能伴隨岩漿入侵與地熱活動。 \\
根據過去在宜蘭平原的熱能研究,該區域顯示出極高的地熱能開發潛力(Chiang et al., 2021)。多項研究指出,區內具有顯著的地溫梯度與熱流值,地下溫度適合發電應用。弧後擴張作用與岩漿入侵的可能性進一步提升了地溫梯度。此外,第四紀沖積層的高滲透性有利於地熱資源的開發,顯示宜蘭平原在再生能源發展中具有重要貢獻,並可降低對進口化石燃料的依賴。 \\
為了探索該區的地熱資源,本研究於2022年7月至2023年1月期間佈設了一組包含81個測站的地震陣列,覆蓋整個宜蘭平原,並在紅柴林地區佈設了186個高密度測站(圖2)。本次大規模測站部署的目標是提供高解析度數據,以建立精確的地下地震速度模型,進而有效識別並利用區內的地熱資源。

\captionsetup{type=figure}
\captionof{figure}{(a) 台灣及其周邊構造的地圖,紅色方框標示出宜蘭平原的位置。
(b) 宜蘭平原,顯示由81個測站(黃色三角形)組成的宜蘭陣列分布情況,紅色方框標示紅柴林地區的位置。
(c) 紅柴林地區,展示由186個測站(倒三角形)組成的高密度地震陣列,蘭陽溪穿過其中,紫色方框標示出台灣第一口深層地熱鑽井的位置。}\label{tbl:nicetablelesstable}
\endgroup


\section{研究動機與背景}

。:,。;。

\begin{tabular}{ll}
What & is \\
this & doing? \\
\end{tabular}

\section{文獻回顧}

,:,。;。
:,。;。\par

\subsection{地質背景}
。:,。;。
\subsection{前人研究}
。:,。;。
\subsection{雷利波橢圓度發展}
。:,。;。
\section{研究內容}

自:,。;。