% !TeX root = ../main.tex

\begin{abstract}
    台灣東北部的宜蘭平原位於沖繩海槽西南端,其特徵為弧後擴張,可能伴隨岩漿入侵與顯著的地熱活動。為探索其地熱資源,本研究於2022年7月至2023年7月期間,進行了一次綜合地震調查,佈設了覆蓋整個宜蘭平原的81個測站陣列,以及位於周圍紅柴林地區的186個高密度測站陣列。研究首先應用水平-垂直(H/V)頻譜比法,透過將H/V振幅頻譜比的峰值頻率轉換為基岩上覆沖積層的厚度,來描繪宜蘭平原的基底幾何結構。結果顯示沖積層厚度向東北方向逐漸加深,並覆蓋於傾斜的中新世基底之上,與先前研究一致。

接下來,我們基於構建的基底幾何,為各測站建立雙層初始速度模型,並應用極化度方法(DOP-E)從連續的環境噪聲數據中提取雷利波橢圓度,進一步進行速度模型反演。採用鄰域算法反演雷利波橢圓度,生成各測站的更精細一維剪切波速度模型,並構建宜蘭平原的半三維速度模型,以探測潛在的地熱/熱液熱源。

本研究建立的微動分析流程是一種非侵入性、高效率的單測站方法,可快速評估裂谷盆地的基底與地熱結構。
\end{abstract}

\begin{abstract*}
    The Ilan Plain in northeastern Taiwan, situated at the southwestern tip of the Okinawa Trough, is characterized by back-arc extension, which may involve magma intrusion and significant geothermal activities. To explore its geothermal resources, a comprehensive seismic survey was conducted using an array of 81 stations covering the entire Ilan Plain and a very dense array of 186 sensors in the surrounding Hongchailin area, deployed from July 2022 to July 2023. This study first utilizes the Horizontal-to-Vertical (H/V) spectral ratio method to delineate the basement geometry of the Ilan Plain, by converting the peak frequency of the H/V amplitude spectral ratio into the thickness of alluvial deposits overlying the bedrock. The results reveal a general northeastward thickening of alluvial layers over a tilting Miocene basement, consistent with previous studies. We then set up two-layer initial velocity models for individual stations based on the constructed basement geometry and apply the degree-of-polarization method (DOP-E) to extract the ellipticity of Rayleigh waves from continuous ambient noise data for further velocity model inversion. The neighborhood algorithm is used to invert the Rayleigh wave ellipticity for finer 1-D shear velocity models at each station to construct a semi-3D velocity model for the Ilan Plain for investigating potential geothermal/hydrothermal heat sources. The established workflow of microtremor analysis promises a non-invasive, efficient single-station approach for rapidly assessing basement and geothermal structures in rifted basins.
    \end{abstract*}