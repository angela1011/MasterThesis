% !TeX root = ./main.tex

% --------------------------------------------------
% 資訊設定(Information Configs)
% --------------------------------------------------

\ntusetup{
  university*   = {National Taiwan University},
  university    = {國立臺灣大學},
  college       = {理學院},
  college*      = {College of Science},
  institute     = {地質科學研究所},
  institute*    = {Department of Geosciences},
  title         = {透過密集陣列微震分析構建台灣北部宜蘭平原高解析Vs速度模型用於地熱探勘},
  title*        = {Constructing High-Resolution Shear-Wave Velocity Model for Geothermal Exploration in the Ilan Plain, Northern Taiwan through Dense-Array Microtremor Analysis },
  author        = {康譯云},
  author*       = {I-Yun Kang},
  ID            = {R12224203},
  advisor       = {黃信樺、吳逸民},
  advisor*      = {Hsin-Hua Huang, Yih-Min Wu},
  % date          = {2025-06-25},         % 若註解掉,則預設為當天
  % oral-date     = {2025-06-25},         % 若註解掉,則預設為當天
  DOI           = {10.5566/NTU2025XXXXX},
  keywords      = {微震分析、地熱探勘、密集陣列、雷利波橢圓度、極化度},
  keywords*     = {Microtremor analysis, HVSR, Basin structure, Geothermal energy, Dense seismic array, Rayleigh waves ellipticity, Degree-of-polarization},
}

% --------------------------------------------------
% 加載套件(Include Packages)
% --------------------------------------------------

\usepackage[sort&compress]{natbib}      % 參考文獻
\usepackage{amsmath, amsthm, amssymb}   % 數學環境
\usepackage{ulem, CJKulem}              % 下劃線、雙下劃線與波浪紋效果
\usepackage{booktabs}                   % 改善表格設置
\usepackage{multirow}                   % 合併儲存格
\usepackage{diagbox}                    % 插入表格反斜線
\usepackage{array}                      % 調整表格高度
\usepackage{longtable}                  % 支援跨頁長表格
\usepackage{paralist}                   % 列表環境


\usepackage{lipsum}                     % 英文亂字
\usepackage{zhlipsum}                   % 中文亂字

% --------------------------------------------------
% 套件設定(Packages Settings)
% --------------------------------------------------
